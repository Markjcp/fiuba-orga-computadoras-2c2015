\documentclass[9pt,a4paper]{article}
\usepackage[utf8]{inputenc}

\usepackage[spanish]{babel}
\usepackage[utf8]{inputenc}
\usepackage{geometry}
\usepackage{url}
\usepackage{amsmath}
\usepackage{graphicx}
\usepackage{listings}
\usepackage{color}
\usepackage{multicol}
\usepackage{fancyhdr}

\pagestyle{fancy}

\definecolor{grey}{rgb}{0.8,0.8,0.8}
\usepackage{listings}

%Configuración de lstlisting:
\lstset{
language=C,
tabsize=4,
basicstyle=\fontsize{11}{13}\ttfamily\footnotesize,
showspaces=false,
showstringspaces=false,
captionpos=b,
breaklines=true
}

\title{6620 - Organización de Computadoras}

\author{Beltrán, Belén \\ (91718) \and Forlenza, Marcos \\ (87237)}

\lhead{}
\rhead{}
\chead{}

\cfoot{\thepage}
\begin{document}
\date{}

\maketitle
\thispagestyle{empty}
\begin{abstract}
El presente trabajo tiene como objetivo la familiarización con las herramientas de software que se utilizarán a lo largo de los siguientes trabajos prácticos, por lo que se lleva a cabo la implementación de un programa que resuelve cierta problemática detallada en los próximos apartados.
\end{abstract}

\tableofcontents
\pagebreak

\section{Introducción}
El programa realizado consta de un computador de autómatas celulares para cierta regla arbitraria. Devuelve como resultado un archivo .pbm con la evolución de este autómata celular para cierta regla determinada.
\par
La implementación del programa se realizará en el lenguaje de programación C. Luego se ejecutará la aplicación sobre una plataforma \textit{NetBSD/MIPS-32} mediante el emulador \textit{GXEmul} \cite{GXEMUL}. 
\bigskip

\section{Compilación}
	
	La herramienta para compilar el código en lenguaje C será el \textit{GCC}.
	\par
	Para automatizar las tareas de compilación se hace uso de la herramienta \textit{GNU Make}. Los Makefiles utilizados para la compilación se incluyen junto al resto de los archivos fuentes del presente trabajo.
\bigskip

\section{Utilización}
	
Veamos ahora la forma en la que debe ser ejecutado el programa implementado en lenguaje C. El resultado de la compilación con ``make'' será un programa ejecutable, de nombre \textit{autcel}, que podrá ser invocado con los siguientes parámetros:
\medskip

\begin{itemize}

\itemsep=2pt \topsep=0pt \partopsep=0pt \parskip=0pt \parsep=0pt
\item \textit{-h}:  Imprime ayuda para la utilización del programa;
\item \textit{-V}:  Imprimer la versión actual del programa;
\item \textit{-o [Path]}: Especifica la ruta del archivo de salida sobre el cual se guarda la evolución generada por la aplicación.

\end{itemize}

Ejemplos: 
\par
{\ttfamily\footnotesize
\$ ./autcel 30 80 inicial -o evolucion}
\par
Calcula la evolución del autómata "Regla 30", en un mundo unidimensional de 80 celdas por 80 iteraciones.
\par
El archivo de salida se llama evolucion.pbm
\bigskip

\section{Resultados Obtenidos}

A continuación se muestran los resultados obtenidos para algunos casos de prueba:
\par


\section{Conclusiones}

Conclusiones

\bigskip

%Referencias
\newpage
\begin{thebibliography}{99}
\bibitem{GXEMUL} The NetBSD project, \url{http://www.netbsd.org/}

\bibitem{GCC} GCC, the GNU Compiler Collection, \url{http://gcc.gnu.org/}

\bibitem{BS} PBM format speciffication, \url{http://netpbm.sourceforge.net/doc/pbm.html}

\bibitem{HEN00} J. L. Hennessy and D. A. Patterson, ``Computer Architecture. A Quantitative
Approach,'' 4th Edition, Morgan Kaufmann Publishers, 2000.

\end{thebibliography}


\end{document}
