\documentclass[9pt,a4paper]{article}
\usepackage[utf8]{inputenc}
%% PAQUETES

% Paquetes generales
\usepackage[margin=2cm, paperwidth=210mm, paperheight=297mm]{geometry}
\usepackage[spanish]{babel}
\usepackage[utf8]{inputenc}
\usepackage{gensymb}

% Paquetes para estilosappendices
\usepackage{textcomp}
\usepackage{setspace}
\usepackage{colortbl}
\usepackage{color}
\usepackage{color}
\usepackage{upquote}
\usepackage{xcolor}
\usepackage{listings}
\usepackage{caption}
\usepackage[T1]{fontenc}
\usepackage[scaled]{beramono}

% Paquetes extras
\usepackage{amssymb}
\usepackage{float}
\usepackage{graphicx}
\usepackage{url}
\usepackage[toc,page]{appendix}

% Paquete Matematica
\usepackage{mathtools}


%% Fin PAQUETES


% Definición de preferencias para la impresión de código fuente.
%% Colores
\definecolor{gray99}{gray}{.99}
\definecolor{gray95}{gray}{.95}
\definecolor{gray75}{gray}{.75}
\definecolor{gray50}{gray}{.50}
\definecolor{keywords_blue}{rgb}{0.13,0.13,1}
\definecolor{comments_green}{rgb}{0,0.5,0}
\definecolor{strings_red}{rgb}{0.9,0,0}

%% Caja de código
\DeclareCaptionFont{white}{\color{white}}
\DeclareCaptionFont{style_labelfont}{\color{black}\textbf}
\DeclareCaptionFont{style_textfont}{\it\color{black}}
\DeclareCaptionFormat{listing}{\colorbox{gray95}{\parbox{16.78cm}{#1#2#3}}}
\captionsetup[lstlisting]{format=listing,labelfont=style_labelfont,textfont=style_textfont}

\lstset{
	aboveskip = {1.5\baselineskip},
	backgroundcolor = \color{gray99},
	basicstyle = \ttfamily\footnotesize,
	breakatwhitespace = true,   
	breaklines = true,
	captionpos = t,
	columns = fixed,
	commentstyle = \color{comments_green},
	escapeinside = {\%*}{*)}, 
	extendedchars = true,
	frame = lines,
	keywordstyle = \color{keywords_blue}\bfseries,
	language = C,                       
	numbers = left,
	numbersep = 5pt,
	numberstyle = \tiny\ttfamily\color{gray50},
	prebreak = \raisebox{0ex}[0ex][0ex]{\ensuremath{\hookleftarrow}},
	rulecolor = \color{gray75},
	showspaces = false,
	showstringspaces = false, 
	showtabs = false,
	stepnumber = 1,
	stringstyle = \color{strings_red},                                    
	tabsize = 2,
	title = \null, % Default value: title=\lstname
	upquote = true,                  
}

%% FIGURAS
\captionsetup[figure]{labelfont=bf,textfont=it}
%% TABLAS
\captionsetup[table]{labelfont=bf,textfont=it}

% COMANDOS

%% Titulo de las cajas de código
\renewcommand{\lstlistingname}{Código}
%% Titulo de las figuras
\renewcommand{\figurename}{Figura}
%% Titulo de las tablas
\renewcommand{\tablename}{Tabla}
%% Referencia a los códigos
\newcommand{\refcode}[1]{\textit{Código \ref{#1}}}
%% Referencia a las imagenes
\newcommand{\refimage}[1]{\textit{Imagen \ref{#1}}}

%% APENDICES
\addto\captionsspanish{
	\renewcommand\seename{Apéndices}
	\renewcommand\appendixname{Apéndices}
	\renewcommand\appendixpagename{Apéndices}
}

\usepackage[spanish]{babel}
\usepackage[utf8]{inputenc}
\usepackage{geometry}
\usepackage{url}
\usepackage{amsmath}
\usepackage{graphicx}
\usepackage{listings}
\usepackage{color}
\usepackage{multicol}
\usepackage{fancyhdr}

\pagestyle{fancy}

\definecolor{grey}{rgb}{0.8,0.8,0.8}
\usepackage{listings}

%Configuración de lstlisting:
\lstset{
language=C,
tabsize=4,
basicstyle=\fontsize{11}{13}\ttfamily\footnotesize,
showspaces=false,
showstringspaces=false,
captionpos=b,
breaklines=true
}

\title{6620 - Organización de Computadoras}

\author{Beltrán, Belén \\ (91718) \and Forlenza, Marcos \\ (87237)}

\lhead{}
\rhead{}
\chead{}

\cfoot{\thepage}
\begin{document}
\date{}

\maketitle
\thispagestyle{empty}
\begin{abstract}
El presente trabajo tiene como objetivo la familiarización con las herramientas de software que se utilizarán a lo largo de los siguientes trabajos prácticos, por lo que se lleva a cabo la implementación de un programa que resuelve cierta problemática detallada en los próximos apartados.
\end{abstract}

\tableofcontents
\pagebreak

\section{Introducción}
El programa realizado consta de un computador de autómatas celulares para cierta regla arbitraria. Devuelve como resultado un archivo .pbm con la evolución de este autómata celular para cierta regla determinada.
\par
La implementación del programa se realizará en el lenguaje de programación C. Luego se ejecutará la aplicación sobre una plataforma \textit{NetBSD/MIPS-32} mediante el emulador \textit{GXEmul} \cite{GXEMUL}. 
\bigskip

\section{Compilación}
	
	La herramienta para compilar el código en lenguaje C será el \textit{GCC}.
	\par
	Para automatizar las tareas de compilación se hace uso de la herramienta \textit{GNU Make}. Los Makefiles utilizados para la compilación se incluyen junto al resto de los archivos fuentes del presente trabajo.
\bigskip

\section{Utilización}
	
Veamos ahora la forma en la que debe ser ejecutado el programa implementado en lenguaje C. El resultado de la compilación con ``make'' será un programa ejecutable, de nombre \textit{autcel}, que podrá ser invocado con los siguientes parámetros:
\medskip

\begin{itemize}

\itemsep=2pt \topsep=0pt \partopsep=0pt \parskip=0pt \parsep=0pt
\item \textit{-h}:  Imprime ayuda para la utilización del programa;
\item \textit{-V}:  Imprimer la versión actual del programa;
\item \textit{-o [Path]}: Especifica la ruta del archivo de salida sobre el cual se guarda la evolución generada por la aplicación.

\end{itemize}

Ejemplos: 
\par
{\ttfamily\footnotesize
\$ ./autcel 30 80 inicial -o evolucion}
\par
Calcula la evolucion del automata Regla 30, en un mundo unidimensional de 80 celdas por 80 iteraciones.
\par
El archivo de salida se llama evolucion.pbm
\bigskip

\section{Resultados Obtenidos}

A continuación se muestran los resultados obtenidos para algunos casos de prueba:
\par

Para el caso de la regla de 94 en 100 pasos vemos lo siguiente:
\medskip

\begin{figure}[H]
	\centering
	\includegraphics[width=0.36\textwidth]{./imgs/regla 94 de 100x100.png}
	\medskip
	\caption{Imágen de la regla 94 aplicada en 100 pasos.}
\end{figure}
\medskip


Para el caso de la regla de 188 en 80 pasos vemos lo siguiente:
\medskip

\begin{figure}[H]
	\centering
	\includegraphics[width=0.36\textwidth]{./imgs/Regla 188 de 80x80.png}
	\medskip
	\caption{Imágen de la regla 188 aplicada en 80 pasos.}
\end{figure}
\medskip


Para el caso de la regla de 127 en 95 pasos vemos lo siguiente:
\medskip

\begin{figure}[H]
	\centering
	\includegraphics[width=0.36\textwidth]{./imgs/Regla 127 de 95x95.png}
	\medskip
	\caption{Imágen de la regla 127 aplicada en 95 pasos.}
\end{figure}
\medskip

\section{Conclusiones}

Las herramientas necesarias para realizar este trabajo resultaron dentro de todo simples de utilizar. Sin embargo, tuvimos y existen todavía problemas con la función próximo() implementada en assembly. 
\par

La integración entre el código en \textit{C} y el \textit{MIPS} no fue tan compleja y de hecho resultó interesante de realizar. 
\par

También tuvimos la necesidad de debuggear el programa hecho en MIPS, ya que al tener errores era necesario chequear el código paso a paso. Esto fue realizado con el \textit{gdb} y fue de muchísima utilidad a la hora de encontrar errores.

\section{Próximas Mejoras}

Se espera para la próxima entrega poder tener implementada correctamente la función próximo en assembly.

\bigskip

%Referencias
\newpage
\begin{thebibliography}{99}
\bibitem{GXEMUL} The NetBSD project, \url{http://www.netbsd.org/}

\bibitem{GCC} GCC, the GNU Compiler Collection, \url{http://gcc.gnu.org/}

\bibitem{BS} PBM format speciffication, \url{http://netpbm.sourceforge.net/doc/pbm.html}

\bibitem{HEN00} J. L. Hennessy and D. A. Patterson, ``Computer Architecture. A Quantitative
Approach,'' 4th Edition, Morgan Kaufmann Publishers, 2000.

\end{thebibliography}


\newpage

% Apendices
\begin{appendices}

\bigskip\bigskip

% Implementación completa en lenguaje C
\section{Implementación completa en lenguaje C}


\subsection{\textit{main.c}. Implementación del main del programa}
% Código
\lstset{ language = C } % Cambiamos el lenguaje para que parsee en C
\lstinputlisting[label=codeMAINcfull,caption=``main.c'']{./src/main.c} 
\bigskip\bigskip

\subsection{\textit{plotter.h}. Declaración del algoritmo Bubblesort}
% Código
\lstset{ language = C } % Cambiamos el lenguaje para que parsee en C
\lstinputlisting[label=codePLOTTERhfull,caption=``plotter.h'']{./src/plotter.h} 
\bigskip\bigskip

\subsection{\textit{plotter.c}. Definición del algoritmo Bubblesort}
% Código
\lstset{ language = C } % Cambiamos el lenguaje para que parsee en C
\lstinputlisting[label=codePLOTTERcfull,caption=``plotter.c'']{./src/plotter.c} 
\bigskip\bigskip

\subsection{\textit{proximo.h}. Declaración del algoritmo Heapsort}
% Código
\lstset{ language = C } % Cambiamos el lenguaje para que parsee en C
\lstinputlisting[label=codePROXIMOhfull,caption=``proximo.h'']{./src/proximo.h} 
\bigskip\bigskip

\subsection{\textit{proximo.c}. Definición del algoritmo Heapsort}
% Código
\lstset{ language = C } % Cambiamos el lenguaje para que parsee en C
\lstinputlisting[label=codePROXIMOcfull,caption=``proximo.c'']{./src/proximo.c} 
\newpage



% Implementación completa en lenguaje C
\section{Implementación completa en lenguaje Assembly MIPS32}

\subsection{\textit{proximo.S}. Definición del algoritmo Heapsort}
% Código
\lstset{ language = [mips]Assembler} % Cambiamos el lenguaje para que parsee en MIPS
\lstinputlisting[label=codePROXIMOsfull,caption=``proximo.S'']{./src/proximo.S} 
\bigskip\bigskip

\end{appendices}

\end{document}
